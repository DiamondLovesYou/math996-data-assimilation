\documentclass[a4paper, 12pt, notitlepage]{report}
\usepackage{mathtools, hyperref}

\usepackage[left=1in, right=1in, top=1in, bottom=1in]{geometry}

\usepackage{titling, bm}
\usepackage{lipsum}

\pretitle{\begin{center}\Huge\bfseries}
\posttitle{\par\end{center}\vskip 0.5em}
\preauthor{\begin{center}\Large\ttfamily}
\postauthor{\end{center}}
\predate{\par\large\centering}
\postdate{\par}

\title{A study of standard, implicit, and optimal particle filters
    applied to a shallow water model (DRAFT, INCOMPLET)}
\author{Richard Diamond \\ \href{mailto:richarddiamond@ku.edu}{richarddiamond@ku.edu}}
\date{\today}

\RequirePackage[style=chem-acs, backend=biber]{biblatex}
\addglobalbib{refs.bib}

\delimitershortfall-1sp
\newcommand\abs[1]{\left|#1\right|}
\newcommand{\iu}{{i\mkern1mu}}

\nonstopmode

\begin{document}
  \maketitle
  \thispagestyle{empty}

  \begin{abstract}
  Typical data assimilation techniques including 3/4DVar and ensemble
  Kalman/particle filters (EnKF/PF) can and do fail on hard nonlinear models due
  to errors in the linearization in 3 and 4DVar or due to particle significance
  collapse as in EnKF/PF. In high dimensional models, the space designated by
  the observations is relatively small. This is hard for a Bayesian filter to
  produce likely samples. Implicit sampling extends particle filters, in much
  the same way as in the optimal filter but to a lesser extent, by making use of
  some of the data during sampling. In this paper, we code and compare standard,
  implicit, and optimal proposal particle filters to a shallow water model.
  \end{abstract}

  \section{Introduction}
  In the geosciences, one is concerned with the state of a system. Even in the
  presence of a perfect model, assimilating sparse observations can be
  challenging. Sequential data assimilation schemes are used to combine
  observations of a physical system state into an estimate of it's current
  state. This state estimate, or analysis, is then used as input to forecasting
  steps, resulting in probability densities.

  \par
  TODO: recall the various discrete time filtering algos.

  \par
  In this paper, we implement the spectral shallow water equations in a spherical
  geometry model with Rossby-Haurwitz Wave number four initial conditions.
  \par
  First we will discuss the model as applied in a spherical setting, the
  Rossby-Haurwitz waves used to test the numerical method, and the spectral . Next we will discuss
  the data assimilation techniques as applied to the model
  \section{Model}
  The shallow water equations are a hyperbolic system of partial differential
  equations. They describe the behavior of a fluid flow assuming pressure
  invariance, derived from depth-integrating the Navier-Stokes equations,
  assuming the horizontal length is much larger than the vertical length.
  The equations are given by:
  \begin{equation} \label{swe}
    \begin{split}
      \frac{\partial u}{\partial t} + \frac{u}{a\ cos\ \theta}\frac{\partial
        u}{\partial \lambda} + \frac{u}{a}\frac{\partial u}{\partial \theta} -
      \frac{tan\ \theta}{a}vu - fv &= -\frac{g}{a\ cos\ \theta}\frac{\partial
        h}{\partial \lambda} \\
      \frac{\partial v}{\partial t} + \frac{v}{a\ cos\ \theta}\frac{\partial
        v}{\partial \lambda} + \frac{v}{a}\frac{\partial v}{\partial \theta} +
      \frac{tan\ \theta}{a}u^2+fu&=-\frac{g}{a}\frac{\partial h}{\partial
        \theta} \\
      \frac{\partial h}{\partial t} + \frac{h}{a\ cos\ \theta}\frac{\partial
        h}{\partial \lambda} + \frac{h}{a}\frac{\partial h}{\partial \theta} +
      \frac{h}{a\ cos\ \theta}\bigg[\frac{\partial u}{\partial \lambda} +
        \frac{\partial \big(cos\ \theta\big)}{\partial \theta}\bigg] &= 0
    \end{split}
  \end{equation}
  where \(g\) is the acceleration due to gravity, \(gh\) is the free surface
  geopotential, \(h\) is the free surface height, \(f=2\Omega sin\, \theta\) is
  the Coriolis parameter, \(\Omega\) is Earth's angular velocity, \(\theta\) is
  the angle of latitude, \(\lambda\) is the longitude, \(a\) is Earth's radius,
  and \(V=u_x+v_y\) is the horizontal velocity vector on the surface of the
  sphere.
  \par
  Rossby waves are a \beta-plane approximation of the barotropic vorticity
  equation to the deformation of Earth from a perfect sphere, and are related to
  large scale waves in the atmosphere. Haurwitz later introduced equivalent
  solutions for the sphere, now called the Rossby-Haurwitz waves, resulting in
  propagating solutions of the non-divergent barotropic vorticity equation on a
  sphere. Rossby-Haurwitz waves are widely used as test cases for numerical
  methods for solving the shallow water equations in spherical geometry
  \cite{williamson1992standard}. It was shown in \cite{thuburn2000numerical}
  that the Rossby-Haurwitz wave number four breaks down into turbulent flows
  after long term numerical integration.
  \par
  The initial velocity field is defined as:
  \begin{equation} \label{ivf}
    \begin{split}
      u &= a\omega\, cos\, \phi + a K\, cos^{r-1}\phi\big(r\, sin^2\,
      \phi-cos^2\, \phi\big)\, cos\big(r\lambda\big) \\
      v &= -a K r\, cos^{r-1}\, \phi\, sin\, \phi\, sin\big(r\lambda\big)
    \end{split}
  \end{equation}
  and the initial height field is defined as:
  \begin{equation} \label{ihf}
    \begin{split}
      h &= h_0 + \frac{a^2}{g}\big[A(\phi)+B(\phi)\,cos(r\lambda) + C(\phi)\,
        cos(2r\lambda)\big]
    \end{split}
  \end{equation}
  with:
  \begin{equation} \label{rhwabc}
    \begin{split}
      A(\phi) &= \frac{\omega}{2}(2\Omega+\omega)\, cos^2\, \phi +
      \frac{1}{4}k^2\,cos^{2r}\,\phi\big[(r+1)\, cos^2\,\phi +
        (2r^2-2r-2)-2r^2\,cos^2\phi\big] \\
      B(\phi) &= \frac{2(\Omega +
        \omega)}{(r+1)(r+2)}\,cos^r\phi\big[(r^2+2r+2)-(r+1)^2\,cos^2\,\phi\big]
      \\
      C(\phi) &= \frac{1}{4}k^2\, cos^{2r}\, \phi\big[(r+1)\, cos^2\,\phi - (r+2)\big]
    \end{split}
  \end{equation}
  where \(h_0\) is the height at the poles, \(r\) is the wave number, \(\omega\) is the
  west-east zonal wind phase velocity, and \(k\) is the wave amplitude.

  \subsection{Discretization by spectral decomposition}
  The spectral decomposition, as given in \cite{jakob1995spectral}, is
  \begin{equation} \label{spectraldecomp}
    \begin{split}
      \phi(\lambda,\mu) &= \sum_{m=-M}^{M}
      \sum_{n=\abs{m}}^{\Re(m)}\phi_{m,n}P_{m,n}(\mu)\exp(\iu m \lambda)
    \end{split}
  \end{equation}
  where \(P_{m,n}(\mu)\) is the Legendre polynomial, \(P_{m,n}(\mu)\exp(\iu m
  \lambda)\) are the spherical harmonic functions, \(M\) is the highest Fourier
  wavenumber in the east-west representation, and \Re(m) is the highest degree
  of the corresponding Legendre polynomials for longitudinal wavenumber \(m\).
  \par
  The coefficients of \ref{spectraldecomp} are determined by:
  \begin{equation} \label{spectraldecompcoeff}
    \begin{split}
      \phi_{m,n} &= \int_{-1}^1 \frac{1}{2\pi} \int_0^{2\pi} \phi(\lambda)
      \exp(-\iu m \lambda) P_{m,n}(\mu)d\lambda d\mu
    \end{split}
  \end{equation}
  The inner integral is computed with a fast Fourier Transform algorithm. The
  outer integral is evaluated using Gaussian quadrature on the transform grid
  \begin{equation} \label{grid}
    \begin{split}
      \phi_{m,n} &= \sum_{j=1}^J\phi_m(\mu_j)P_{m,n}(\mu_j)w_j
    \end{split}
  \end{equation}
  where the \(\mu_j\)s are at the Gaussian latitudes \(\theta_j\), which
  correspond to the J roots of the Legendre polynomial \(P_j(sin\,\theta_j)=0\).
  \par
  The size of the grid is determined to prevent the aliased representation of
  quadratic terms:
  \begin{equation} \label{gridsize}
    \begin{split}
      TODO: causes\ terrible\ pixelation
      %I \geq 3\Re(m)+1 \\
      %J \geq \frac{3 max \Re(m) +1}{2}
    \end{split}
  \end{equation}
  where \(N\) is the highest wavenumber from the latitudinal Legendre
  representation.
  \section{Assimilation}
  The dynamical system of the form:
  \begin{equation} \label{idosde}
    \begin{split}
      d\bm{x}=f(\bm{x},t)+g(\bm{x},t)dw
    \end{split}
  \end{equation}
  where \(x\) is an \(m\)-dimensional vector and \(f\) is an \(m\)-dimensional
  vector function.
  \section{Experiment}
  For this paper's experiment, we generate the truth from the model with initial
  conditions propagated forward in time by the model without noise (ie with
  deterministic dynamics).
  \nocite{*}
  \printbibliography

\end{document}
